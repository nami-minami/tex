\documentclass[uplatex, dvipdfmx, 10pt]{jarticle}
    \usepackage{siunitx}
    \usepackage{mhchem}
    \title{{\huge Bacteria cellrose}}
    \author{南川新明}
        \begin{document}
        
            \begin{titlepage}
                \begin{center}
                    \textgt{\Large 令和4年度}

                    \vspace{10truept}

                    \textgt{\Large  機能材料実験}

                    \vspace*{180truept}

                    \textgt{\Huge 植物セルロースを用いたパルプの調製および紙抄き}

                    \vspace{100truept}
                    
                    \textgt{応用化学・生物系4年}

                    \vspace{10truept}

                    \textgt{\Large 34番 南川新明}

                    \vspace{70truept}
                    
                    \textgt{\Large 共同実験者}

                    \vspace{10truept}

                    \textgt{共同実験者}
                \end{center}
            \end{titlepage}
        
        \section*{要旨}

        \section*{緒言}
            
            セルロースはグルコースが $\beta-1,4-$グリコシド結合で多数連なった高分子である.
            これは植物細胞の主成分であり,水と二酸化炭素から合成される.セルロースは
            地球上で最も豊富な高分子であり,植物由来であるから再生産が容易なため,
            石油に代わるカーボンニュートラルな材料である.セルロースは現在紙や繊維の材料として
            大量に使用されている.

        \section*{操作}
            
           \subsection*{アシの蒸解とパルプの漂白}
                
                ステンレスビーカーにアシ $\SI{25}{g}$ ,アシが浸る程度の
                $\SI{5}{\%}$ 水酸化ナトリウム水溶液を入れ,$90-\SI{95}{\celsius}$ に
                50分間保持した.こうして得たパルプをよく水洗し,1週間水に浸して静置した.

                ビーカーにパルプとパルプが浸る程度の16倍に薄めた漂白剤を入れ,
                $\SI{55}{\celsius}$ に保持した恒温槽中で攪拌しつつ30分間漂白した.

            \subsection*{抄紙}

                ミキサーにパルプとパルプが浸る程度の水を入れ,30秒程度回転した.これを紙漉き器に
                入れて圧搾し,水分を布で除いた後にアイロンで乾燥して紙を得た.


        \section*{結果}

        \section*{考察}

        \section*{参考文献}

        
    \end{document}
